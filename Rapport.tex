\documentclass{report}

\usepackage[utf8x]{inputenc}
\usepackage[T1]{fontenc}
\usepackage[french]{babel}
\usepackage{lmodern}
\usepackage{amsmath}
\usepackage{amssymb}
\usepackage{textcomp}
\usepackage[x11names]{xcolor}
\usepackage{tikz}
\usepackage[top=2cm, bottom=2cm, left=2cm, right=2cm]{geometry}
\usepackage{ulem}
\usepackage{enumitem}
\usepackage{fourier-orns}
\usepackage{graphicx}
\usepackage[explicit]{titlesec}
\usepackage{gensymb}
\usepackage{fancyhdr}
\usepackage{titling}
\usepackage{array,multirow,makecell}
\usepackage{algorithmic}

\usetikzlibrary{arrows, positioning, fit, shapes, decorations.pathreplacing, calc}

\title{}\let\mytitle\thetitle
\author{Karim \textsc{Balde}\\Yasmine \textsc{Kertous}\\Thibault \textsc{Lasou}}
\date\today

\newcommand{\tikzmark}[1]{\tikz[overlay,remember picture] \node[text height=6pt] (#1) {};}

\newcommand\reduline{\bgroup\markoverwith
{\textcolor{red}{\rule[-0.5ex]{2pt}{0.4pt}}}\ULon}
\newcommand\greenuline{\bgroup\markoverwith
{\textcolor{green}{\rule[-0.5ex]{2pt}{0.4pt}}}\ULon}
\newcommand\blueuline{\bgroup\markoverwith
{\textcolor{blue}{\rule[-0.5ex]{2pt}{0.4pt}}}\ULon}

\titleformat{\chapter}[display]{\LARGE\bfseries}{\centering {{\mbox{\chaptertitlename \hspace{5pt}\thechapter}}}}{0pt}{\smallskip\centering #1}
\titleformat{\section}[hang]{\Large\bfseries}{\thesection}{10pt}{\reduline{#1}}
\titleformat{\subsection}[hang]{\large\bf}{\thesubsection}{10pt}{\uline{#1}}

\fancypagestyle{polytech}
{
	\fancyhf{}
	\lhead{\includegraphics[height=1cm]{POLYTECH_PARIS-SUD_RVB.jpg}}
	\rhead{\theauthor}
	\chead{\centering{{\bf \Large{\mytitle}}}}
	\cfoot{Page \thepage}
	\lfoot{\today}
}

\fancypagestyle{plain}
{
	\fancypagestyle{polytech}
}

\footskip = 40pt
\headheight = 35pt
\textheight = 650pt

\newcommand{\diff}{\mathop{}\mathopen{}\mathrm{d}}
\newcommand{\R}{\mathop{}\mathopen{}\mathbb{R}}
\renewcommand*\thesection{\arabic{section}}

\begin{document}
\pagestyle{polytech}
\chapter*{Projet ET4 - TAL\\CookBot}
\tableofcontents
\section*{Introduction}

Un chatbot, aussi appelé \og agent conversationnel \fg{}, est un programme informatique pouvant dialoguer avec un individu ou utilisateur par le biais d'un service de conversations automatisées vocal ou textuel. Cet outil est aujourd 'hui très utilisé sur internet par les services clients ou de commerçants en ligne à travers la messagerie instantanée.\\

À l'origine, le chatbot fonctionne en s'appuyant sur une base de données de questions-réponses qui sont déclenchées en fonction de certains mots-clés repérés dans la conversation. Mais les progrès de l'intelligence artificielle, plus précisément de l'apprentissage automatique, ont permis de créer des agents conversationnels beaucoup plus évolués dotés d'un système d'analyse du langage naturel très performant qui  permet de plus en plus  d'analyser et comprendre les messages et qui sont capables de s'améliorer au fur et à mesure de leur utilisation.\\


Aujourd'hui l'utilisation des chatbots s'est considérablement étendue, ils sont capables de répondre à des besoins ponctuels comme :
\begin{itemize}
\item réserver un moyen de transport, un hébergement;
\item commander un repas;
\item rechercher un produit;
\item demander la localisation GPS;
\item répondre à une question technique;...
\end{itemize}
Pour notre projet on a choisi de réaliser un \og Cookbot \fg{} qui est une sorte d'aide culinaire. On a choisi ce thème car l’alimentation est importante pour avoir une vie saine. Se nourrir C'est aussi un acte social, émotif et vital. On se nourrit le plus souvent parce qu'on a faim; parfois aussi pour combler un manque affectif ou pour faire plaisir à ses hôtes… Néanmoins ce n'est pas toujours facile de cuisiner c'est pour cela qu'on propose ce Cookbot qui va faciliter la vie de ses utilisateurs.\\

\section{Description du projet}
\subsection{Objectifs du projet}
Le but de notre projet était donc de créer un chatbot qui serait comme une aide culinaire. Sa fonction principale est de conseiller un menu ou un plat à l'utilisateur en fonction des ingrédients disponibles dans son garde-manger. 
Le cookbot sera relié à une base de données de recettes et doté d'autres fonctionnalités : 
\begin{itemize}
\item afficher une recette ( ingrédients et préparation) à partir de son titre ou intitulé;
\item afficher une/des recette(s) à partir d'une culture(Japonais/italien);
\item indiquer les quantités des ingrédients d'une recette à partir du nombre de convives;
\item répertorier des recettes favorites;
\item répertorier les recettes/aliments non souhaités (chou-fleur, porc,…);
\item faire un historique des recettes déjà recherchées;
\item planifier des événements (anniversaire, invitation, …);
\item suggérer des recettes (avec des fruits et légumes de saison par exemple);
\item comprendre de nouveau paramètres ( aujourd'hui je n'ai pas envie de riz);
\item émettre des suggestions (tu ne devrais pas manger 3 fois par semaine de la pizza);
\item possibilité d'ajouter des recettes;
\end{itemize}
\subsection{Organisation du travail}
Pour réaliser ce projet, on a tout d'abord établi un cahier des charges ou on a spécifié les objectifs du  projet en le découpant en tâches/objectifs puis en les triant selon un ordre de priorité:
\subsubsection{Priorité 0 :  Commencer une conversation } Permettre à l'utilisateur de dialoguer à travers 3 modes : 
\begin{itemize}
\item mode naïf : \og (mots ou expressions qui expriment la faim/l'envie de se nourrir)[ingrédients] [nombre de personnes]\fg{}
\item mode orienté : \og (mots ou expressions qui expriment la faim/l'envie de se nourrir)[type de nourriture] [culture] [nombre de personnes]\fg{}
exemples : Je voudrais manger italien aujourd'hui, Je veux un dessert pour 3 personnes
\item  mode précis : \og ( mots ou expressions qui expriment le besoin)[nom de la recette]\fg{}
\end{itemize}
\subsubsection{Priorité 1 : Création et organisation de la base de données}
\begin{itemize}
\item Base de données qui contient des recettes avec des intitulés, une liste d'ingrédients, le temps de préparation, le nombre de personnes,…
\item Pour remplir la base de données : utilisation d'une bibliothèque Python qui sert à parser du code html tiré à partir d'un site internet de recettes (www.marmiton.org).
\item Ajouter des options pour par exemple ajouter des recettes. 
\end{itemize}
\subsubsection{Priorité 2 :  Intégrer une sorte de mémoire} ajouter des fonctionnalités pour permettre à l'utilisateur de :
\begin{itemize}
\item Avoir la possibilité de créer une liste d'ingrédients bannis. Par exemple \og Je n'aime pas les brocolis \fg{};
\item Avoir un historique de recettes ;
\item Prévoir des événements. Par exemple : \og Mes parents mangent avec moi Samedi et j'aimerais un dessert... \fg{} […] Samedi : \og Voici le dessert que vous avez planifié \fg{}
\end{itemize}
\subsubsection{Priorité 3 : Avoir des conversations } c'est un mode beaucoup plus avancé qui pourra prendre part à des conversations et qui n'est pas seulement une machine qui propose des réponses répertoriées. On ajoutera des fonctionnalités plus subtiles qui pourront par exemple de :
\begin{itemize}
\item Suggérer des recettes ( de saison par exemple)
\item Comprendre de nouveaux paramètres. Par exemple : \og aujourd'hui, je ne veux pas de brocolli. \fg{}
\item Exprimer des jugements. \og Tu ne devrais pas manger de la pizza 3 fois par jour.\fg{}
\end{itemize}
\section{Bilan des réalisations}
\subsection{Travail réalisé}

Tout d'abord, on est parti sur l'extraction de texte à partir de pages HTML qu'on a téléchargées sur le site Marmiton qui est un site qui propose des recettes de cuisine avec une explication sur leur préparation. On s'est vite rendu compte que les informations dont on avait besoin (nom de la recette, temps de préparation et de cuisson, ingrédients, etc.) n'étaient pas représentées de la même façon. Pour deux recettes différentes on trouvait parfois ces informations dans des balises différentes. Ajouté à cela le temps que prenait la lecture et la récupération d'un seul fichier HTML pour une recette. Temps qui était loin d'être négligeable si on essayait de parser plusieurs fichiers. On a donc trouvé une autre solution qui allait nous permettre de récupérer les informations des recettes, les représenter de façon plus efficace. On a donc récupéré directement à partir du site web sans examiner les balises HTML les informations qui nous intéressaient. Une fois ces informations récupérées, on les a structuré dans un fichier XML pour rendre leur utilisation exploitable par un script python qui allait parser ce fichier et stocker son contenu dans des structures de données appropriées. La pluralité des informations que contenait une recette a orienté notre choix pour une représentation des données en forme d'objets avec pour champs les caractéristiques d'une recette (nom, ingrédients, ...). On a donc parsé le fichier XML et stocké les informations de chaque recette dans un objet qu'on a créé. Au final on a une liste d'objets qui contiennent toutes les recettes du fichier qu'on a parsé. \\ \\
Pour la partie analyse on a téléchargé la bibliothèque treetaggerwrapper qui nous a permis de tagger les mots de la phrase écrite par l'utilisateur. On a créé une classe "Request" qui définit la demande de l'utilisateur, elle est caractérisée par : 
\begin{itemize}
\item son nom
\item son type
\item la liste des ingrédients disponibles
\item sa culture
\item son temps de préparation 
\item le nombre de personnes qui peuvent déguster la recette
\end{itemize}
Un objet de type "Request" est créé et le contenu de ses champs seraient déduits des entrées de l'utilisateur. \\ \\ 
On aurait ensuite voulu effectuer une recherche dans la liste des recettes et en fonction du nombre de résultats qui conviennent on aurait demandé des précisions jusqu'à ce qu'il reste 2 ou 3 recettes qu'on aurait affiché à la fin. Or on a malheureusement pas pu effectuer cette partie faute de temps.


\subsection{Contribution de chacun des membres}
Karim s'est essentiellement occupé de parser les fichiers HTML et de créer les fichiers XML. Quant à Thibault, il s'est d'abord occupé de rédiger le cahier des charges, de spécifier les tâches et les prioriser, il s'est ensuite occupé de la partie analyse et enfin Yasmine s'est principalement occupée de la rédaction du rapport.
\section{Améliorations possibles}

Avant d'effectuer d'éventuelles améliorations, il serait d'abord intéressant de compléter la programmation des fonctionnalités enoncées dans le cahier des charges.\\
Ensuite, optimiser le stockage des recettes. En effet, dans la configuration actuelle on a choisi, par manque de temps, de stocker les recettes dans des fichiers XML, cette solution s'avère longue à l'exécution.

\section*{Conclusion}À terme, les chatbots pourraient même se substituer aux applications mobiles en proposant un mode d'interaction beaucoup plus simple et intuitif. Les champs d'application des chatbots sont potentiellement illimités et leur perfectionnement est désormais corrélé aux progrès de l'intelligence artificielle.\\

Mener de bout en bout un projet conséquent avec des règles préétablies nous a permis à la fois de mettre en pratique ce que nous avons appris en cours et de renforcer notre travail en groupe.

\end{document}
