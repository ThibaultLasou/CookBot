\documentclass{report}

\usepackage[utf8]{inputenc}
\usepackage[T1]{fontenc}
\usepackage[french]{babel}
\usepackage{lmodern}
\usepackage{amsmath}
\usepackage{amssymb}
\usepackage{textcomp}
\usepackage[x11names]{xcolor}
\usepackage{tikz}
\usepackage[top=2cm, bottom=2cm, left=2cm, right=2cm]{geometry}
\usepackage{ulem}
\usepackage{enumitem}
\usepackage{fourier-orns}
\usepackage{graphicx}
\usepackage[explicit]{titlesec}
\usepackage{gensymb}
\usepackage{fancyhdr}
\usepackage{titling}
\usepackage{array,multirow,makecell}
\usepackage{algorithmic}

\usetikzlibrary{arrows, positioning, fit, shapes, decorations.pathreplacing, calc}

\title{}\let\mytitle\thetitle
\author{Karim \textsc{Balde}\\Yasmine \textsc{Kertous}\\Thibault \textsc{Lasou}}
\date\today

\newcommand{\tikzmark}[1]{\tikz[overlay,remember picture] \node[text height=6pt] (#1) {};}

\newcommand\reduline{\bgroup\markoverwith
{\textcolor{red}{\rule[-0.5ex]{2pt}{0.4pt}}}\ULon}
\newcommand\greenuline{\bgroup\markoverwith
{\textcolor{green}{\rule[-0.5ex]{2pt}{0.4pt}}}\ULon}
\newcommand\blueuline{\bgroup\markoverwith
{\textcolor{blue}{\rule[-0.5ex]{2pt}{0.4pt}}}\ULon}

\titleformat{\chapter}[display]{\LARGE\bfseries}{\centering {{\mbox{\chaptertitlename \hspace{5pt}\thechapter}}}}{0pt}{\smallskip\centering #1}
\titleformat{\section}[hang]{\Large\bfseries}{\thesection}{10pt}{\reduline{#1}}
\titleformat{\subsection}[hang]{\large\bf}{\thesubsection}{10pt}{\uline{#1}}

\fancypagestyle{polytech}
{
	\fancyhf{}
	\lhead{\includegraphics[height=1cm]{POLYTECH_PARIS-SUD_RVB.jpg}}
	\rhead{\theauthor}
	\chead{\centering{{\bf \Large{\mytitle}}}}
	\cfoot{Page \thepage}
	\lfoot{\today}
}

\fancypagestyle{plain}
{
	\fancypagestyle{polytech}
}

\footskip = 40pt
\headheight = 35pt
\textheight = 650pt

\newcommand{\diff}{\mathop{}\mathopen{}\mathrm{d}}
\newcommand{\R}{\mathop{}\mathopen{}\mathbb{R}}

\begin{document}
\pagestyle{polytech}
\chapter*{Projet ET4 - TAL\\CookBot}
\section*{Introduction}

Un chatbot, aussi appelé « agent conversationnel », est un programme informatique pouvant dialoguer avec un individu ou utilisateur par le biais d'un service de conversations automatisées vocal ou textuel. Cet outil est aujourd 'hui très utilisé sur internet par les services clients ou de commerçants en ligne à travers la messagerie instantanée.\\

À l'origine, le chatbot fonctionne en s'appuyant sur une base de données de questions-réponses qui sont déclenchées en fonction de certains mots-clés repérés dans la conversation. Mais les progrès de l'intelligence artificielle, plus précisément de l'apprentissage automatique, ont permis de créer des agents conversationnels beaucoup plus évolués dotés d'un système d'analyse du langage naturel très performant qui  permet de plus en plus « d'analyser » et « comprendre » les messages et qui sont capables de s'améliorer au fur et à mesure de leur utilisation.\\


Aujourd'hui l'utilisation des chatbots s'est considérablement étendue, ils sont capables de répondre à des besoins ponctuels comme :
\begin{itemize}
\item réserver un moyen de transport, un hébergement;
\item commander un repas;
\item rechercher un produit;
\item demander la localisation GPS;
\item répondre à une question technique;...
\end{itemize}
Pour notre projet on a choisi de réaliser un « Cookbot » qui est une sorte d'aide culinaire. On a choisi ce thème car l’alimentation est importante pour avoir une vie saine. Se nourrir C'est aussi un acte social, émotif et vital. On se nourrit le plus souvent parce qu'on a faim; parfois aussi pour combler un manque affectif ou pour faire plaisir à ses hôtes… Néanmoins ce n'est pas toujours facile de cuisiner c'est pour cela qu'on propose ce Cookbot qui va faciliter la vie de ses utilisateurs.\\

\section{Description du projet}
\subsection{Objectifs du projet}
Le but de notre projet était donc de créer un chatbot qui serait comme une aide culinaire. Sa fonction principale est de conseiller un menu ou un plat à l'utilisateur en fonction des ingrédients disponibles dans son garde-manger. 
Le cookbot sera relié à une base de données de recettes et doté d'autres fonctionnalités : 
\begin{itemize}
\item afficher une recette ( ingrédients et préparation) à partir de son titre ou intitulé;
\item afficher une/des recette(s) à partir d'une culture(Japonais/italien);
\item indiquer les quantités des ingrédients d'une recette à partir du nombre de convives;
\item répertorier des recettes « favorites »;
\item répertorier les recettes/aliments non souhaités (chou-fleur, porc,…);
\item faire un historique des recettes déjà recherchées;
\item planifier des événements (anniversaire, invitation, …);
\item suggérer des recettes (avec des fruits et légumes de saison par exemple);
\item comprendre de nouveau paramètres ( aujourd'hui je n'ai pas envie de riz);
\item émettre des suggestions (tu ne devrais pas manger 3 fois par semaine de la pizza);
\item possibilité d'ajouter des recettes;
\end{itemize}
\subsection{Organisation du travail}

\section{Accopmlissements}
\section{Améliorations possibles}
\section{Contribution de chacun des membres}
\section*{Conclusion}

\end{document}
